Per compilare il progetto è necessario utilizzare Visual Studio \href{https://www.visualstudio.com/it/downloads/}{\tt https\+://www.\+visualstudio.\+com/it/downloads/} (in qualsiasi versione da Community ad enterprise, escluso Visual Studio Code!). Nel programma di aggiunta strumenti e funzionalità è necessario aggiungere all\textquotesingle{}installazione\+:
\begin{DoxyItemize}
\item Sviluppo per desktop .net
\item Sviluppo asp.\+N\+ET e web
\item Sviluppo di Azure
\end{DoxyItemize}

In alternativa è possibile compilare il progetto con gli ide C\# alternativi Xamarin Studio oppure Sharp\+Develop ma potrebbero richiedere modifiche alla struttura interna dei file di progetto e non è garantito il supporto completo delle funzionalità.

Il deploy del progetto compilato viene periodicamente effettuato sul seguente sito\+: \href{http://brewday.cu.cc/}{\tt http\+://brewday.\+cu.\+cc/}

Durante lo sviluppo il sito non rifletterà immediatamente tutti i cambiamenti effettuati sui sorgenti, alla consegna del progetto sarà la versione finale. 